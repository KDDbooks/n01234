
\chapter{まえがき}

信号処理とは,電気回路などの知識をベースとして,音声・画像処理のために大切な知識として位置づけられている.その知識を持ってフィルタの設計などを効率よく行うことができる.

ところで,信号処理という枠組みの科目自体は,電子系もしくは情報系ではほぼ必須アイテムのひとつとされて久しい.また,大学生向けのテキストとしては従来良書と分類されるものも非常に多い.しかしながら,昨今,大学における入試形態の多様化や,カリキュラムの多様性も相まって,従来のテキストで講義を行うための数学的前提条件が十分に整っていない状態であることが非常に多いという声を多く聞くようになってきた.

このことから,本書では,従来の信号処理のテキストにおける内容をできるだけ平易化することだけでなく,必要と思われる数学的なアプローチも随所に記載をして,理解しやすさに重きをおいた.したがって,高等学校で数学III程度の内容がある程度理解できるようであれば,十分に読み進められるように配慮していており,その補完として第2章において数学的な扱いをあえて示している.

また,ディジタルフィルタだけでなく,実用上組み合わせて使用することから知っておくほうがよいと考えるアナログフィルタすなわち電気回路で構成される高域フィルタ,低域フィルタなどについても説明した.最終章では画像のディジタル処理の一部についても紹介し,昨今,多く出回っているアプリケーションプログラムにおける処理のあり方についても述べた.これらを通じて,信号処理をすることにより,ユーザにとって都合の良い信号を抽出し,パターン認識や各種メディア信号処理をすることが理解されれば幸いである.

もちろん,本書で学習したならば,従来より存在する良書により,エンジニアとして必要なさらに高度な知識や教養を身につけて,社会に羽ばたいてほしいと願っている.

さて,授業を行うにあたっては教員の裁量を最優先することは大切なことと承知しているが,もし,本書を用いて90分授業を15週かけて実施する場合には,第\ref{chapter:2}章,第\ref{chapter:ch-2}章,第\ref{chapter:fft}章を2回の授業に分割して授業を行うことを想定している.特に第2章の扱いは受講者の習熟度にあわせて適宜調整されるものと考えている.

著者の浅学非才を顧みず,伝統的な良書への橋渡しのつもりで執筆したが,自分で課題を見つけて回路を実装したり,アプリケーションプログラムを作成したり,さらに深い学びを得るための準備として参考になれば幸いである.
\begin{flushright}
2022年2月\\田中 賢一
\end{flushright}

